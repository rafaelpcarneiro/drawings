% vim: foldmethod=marker: 
\documentclass[11pt]{standalone}

%|--- Packages and Configs {{{1
\usepackage[utf8]{inputenc}
\usepackage[T1]{fontenc}    
\usepackage[brazil]{babel}

\usepackage{amsmath}
\usepackage{amsfonts}
\usepackage{amssymb}
\usepackage{amsthm}
\usepackage{mathrsfs}

\usepackage{mathtools}

\usepackage{bbm} %% \mathbbm{1} gives you the identity function symbol 1
\usepackage{bm} 

%%% allows you to insert many figures indexed by (a), (b), ... on a figure environment
\usepackage{float} 
\usepackage[caption = false]{subfig}

\usepackage{tikz}
\usetikzlibrary{snakes} %% produces curly arrows on tikz
\usetikzlibrary{matrix} %% for commutative diagrams
\usetikzlibrary{arrows}

\usetikzlibrary{shadings}
\definecolor{azulClaro}{RGB}{65,105,225}
\definecolor{azulEscuro}{RGB}{0,0,51}

\usepackage{ifthen} %% gives if conditionals when using newcommand

%% Garamond fonts 
\usepackage{ebgaramond}
\usepackage[ugm]{mathdesign}
% 1}}}

\begin{document}

\begin{tikzpicture}
    
    \tikzstyle pacman=[color=red, ultra thick, fill=yellow]
    \tikzstyle background=[fill=azulEscuro, opacity=0.95]
    \tikzstyle whiteBalls=[fill=yellow]

    \draw[background] (2,-4) rectangle (35,10);

    \draw[pacman] (5,4) -- +(45:2) arc (45:315:2) -- cycle;
    \draw (6,7) node[scale=7] {?};

    \foreach \x in {9,12,...,22}{
        \draw[whiteBalls] (\x, 4) circle (0.5cm);
    }


    %|--- Toro {{{1
    \begin{scope}[scale=0.4, shift={(70,17)}]
        \begin{scope}[xscale=6, yscale=1] %toro
            \foreach \angle in {90, 89, ..., -90}{
                
                \pgfmathsetmacro\x{ cos(\angle)  }
                \pgfmathsetmacro\y{ 2.3*sin(\angle)  }
                \pgfmathsetmacro\intensidadeB{ max(90*cos(\angle/0.8), 49)  }
                \pgfmathsetmacro\intensidadeA{ 100 - max(80*cos(\angle/0.85),10 )  }

                \fill[azulClaro!\intensidadeA!black!\intensidadeB]
                    (\x,\y) ellipse (0.5cm and 2.1cm);

                \fill[azulClaro!\intensidadeA!black!\intensidadeB]
                    (-\x,\y) ellipse (0.5cm and 2.1cm);
            }

            \pgfmathsetmacro\x{ cos(90)  }
            \pgfmathsetmacro\y{ 2.3*sin(-90)  }
            \pgfmathsetmacro\intensidadeB{ max(90*cos(-90/0.8), 49)  }
            \pgfmathsetmacro\intensidadeA{ 100 - max(80*cos(-90/0.85),10 )  }
                                                
            \shade[
                shading=ball,
                outer color = azulClaro!\intensidadeA!black!\intensidadeB,
                inner color = azulClaro!20
            ]
                (\x,\y) ellipse (0.5 cm and 2.1cm);
        \end{scope} %Toro
        
        \begin{scope}[rotate=-90, shift={(-2,5)}]
            \draw[<->, ultra thick, line width = 2pt]
                (2,5) .. controls (6,10) and (14,10) ..  (18,5)
                node[midway, right, scale=2 ] {$\Phi$};
        \end{scope}

        \begin{scope}[shift={(-20,-14)}] % [0,2pi]^2
            \def\mySquare{(14,-4) rectangle (26, 4.5)}

            \draw[very thick, line width = 2pt, fill=gray!70] \mySquare;

            \begin{scope}[even odd rule]
                \clip \mySquare;
                \foreach \x in {0,1, ..., 26}{
                        \draw (\x,-4) -- + (10,10);
                }
            \end{scope}
            
            \draw[ultra thick] (20,-6) node[scale=1.6] {$\bm{[0,2\pi] \times [0,2\pi]}$};


            \draw[->, thick, line width=2pt] (14,-4) -- (20,-4);
            \draw[->, thick, line width=2pt] (14,4.5) -- (20,4.5);
            \draw[->, thick, line width=2pt] (14,-4) -- (14,0.2);
            \draw[->, thick, line width=2pt] (26,-4) -- (26,0.2);
            
            %\draw (14,-4)  node[below left ,scale=1.8]  {$\mathbf{a_0}$};
            %\draw (14,4.5) node[above left ,scale=1.8]  {$\mathbf{a_0}$};
            %\draw (26,-4)  node[below right,scale=1.8] {$\mathbf{a_0}$};
            %\draw (26,4.5) node[above right,scale=1.8] {$\mathbf{a_0}$};

            \draw[fill = black]
                (20,12) circle (0.2cm) node[right,scale=1.8] {$\mathbf{a_0}$};
        \end{scope}% [0,2pi]^2
    \end{scope}
    %|--- END TORO 1}}}

    %%%  diagrama {{{1
    \begin{scope}[shift={(14,-1)}]
        \tikzstyle{every node}=[very thick, scale=2.3]
        \matrix [matrix of math nodes, column sep=3em, row sep=1em]
        {
            |(0C)| (0)      & |(C0)|  C_0    & |(C1)| C_1 &
            |(C2)| C_2      & |(CD)|  \cdots & |(CN)| C_n &
            |(CN1)| C_{n+1} & |(CDD)| \cdots\\   
            %
                            &                 &            &
                            & |(aa)| \cdots   &            &
                            & |(bb)| \cdots\\ 
            %
            |(0D)| (0)      & |(D0)| D_0      & |(D1)| D_1 &
            |(D2)| D_2      & |(DD)| \cdots   & |(DN)| D_n & 
            |(DN1)| D_{n+1} & |(DDD)| \cdots\\   
        };
        \tikzstyle{every node} = [midway, above, scale=1.5]
        \draw[ultra thick, <-] (0C.east) -- node {$\bm{\partial}_0 = 0$} (0C.east -| C0.west);
        \draw[ultra thick, <-] (C0.east) -- node {$\bm{\partial}_1$} (C0.east -| C1.west);
        \draw[ultra thick, <-] (C1.east) -- node {$\bm{\partial}_2$} (C1.east -| C2.west);
        \draw[ultra thick, <-] (C2.east) -- node {$\bm{\partial}_3$} (C2.east -| CD.west);
        \draw[ultra thick, <-] (CD.east) -- node {$\bm{\partial}_n$} (CD.east -| CN.west);
        \draw[ultra thick, <-] (CN.east) -- node {$\bm{\partial}_{n+1}$} (CN.east -| CN1.west);
        \draw[ultra thick, <-] (CN1.east) --  (CN1.east -| CDD.west);

        \draw[ultra thick, <-] (0D.east) -- node {$\bm{\partial}_0 = 0$} (0D.east -| D0.west);
        \draw[ultra thick, <-] (D0.east) -- node {$\bm{\partial}_1$} (D0.east -| D1.west);
        \draw[ultra thick, <-] (D1.east) -- node {$\bm{\partial}_2$} (D1.east -| D2.west);
        \draw[ultra thick, <-] (D2.east) -- node {$\bm{\partial}_3$} (D2.east -| DD.west);
        \draw[ultra thick, <-] (DD.east) -- node {$\bm{\partial}_n$} (DD.east -| DN.west);
        \draw[ultra thick, <-] (DN.east) -- node {$\bm{\partial}_{n+1}$} (DN.east -| DN1.west);
        \draw[ultra thick, <-] (DN1.east) --  (DN1.east -| DDD.west);

        %\drawultra [thick, ->] (0C) -- (0D) node[left] {$f_0$};
        \draw[ultra thick, ->] (C0) -- (D0) node[left] {$f_0$};
        \draw[ultra thick, ->] (C1) -- (D1) node[left] {$f_1$};
        \draw[ultra thick, ->] (C2) -- (D2) node[left] {$f_2$};
        \draw[ultra thick, ->] (CN) -- (DN) node[left] {$f_n$};
        \draw[ultra thick, ->] (CN1) -- (DN1) node[left] {$f_{n+1}$};
    \end{scope}
    %%% 1}}}
\end{tikzpicture}

\end{document}


